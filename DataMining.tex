\documentclass[a4paper,12pt]{article}
\usepackage[T1]{fontenc}
\usepackage[utf8]{inputenc}
\usepackage[ngerman]{babel}
\usepackage[pdftex]{hyperref}
\usepackage{floatflt}
\usepackage{graphicx}
\usepackage{tabularx}
\usepackage{xcolor}
\usepackage{framed}
\usepackage{changes}
\usepackage{tikz}
\usepackage{capt-of}
\usepackage{diagbox}
\usetikzlibrary{arrows.meta}
\usepackage[thmmarks, thref]{ntheorem}
\usepackage{amsmath, amssymb}
\usepackage{eurosym}
%\usepackage[amsmath, thmmarks, thref]{ntheorem}
\usepackage{verbatim}
\usepackage{fancyhdr}
\usepackage[hang]{footmisc}
\renewcommand{\subsectionmark}[\thepage]{\markright{{#1}}}
\textheight220mm
\headheight 28pt
\textwidth160mm
\oddsidemargin0mm
\evensidemargin0mm
\topmargin0mm
\pagestyle{headings}
\fancyhead[R]{\sectionmark}
\newcommand{\F}{\mathbb F}
\newcommand{\f}{\mathbb f}
\newcommand{\K}{\mathbb K}
\newcommand{\R}{\mathcal R}
\newcommand{\N}{\mathbb N}
\newcommand{\Z}{\mathbb Z}
\newcommand{\Q}{\mathbb Q}
\newcommand{\A}{\mathcal A}
\newcommand{\C}{\mathcal C}
\newcommand{\D}{\mathcal D}
\newcommand{\X}{\mathcal X}
\newcommand{\Y}{\mathcal Y}
\newcommand{\xL}{\mathcal L}
\newcommand{\xP}{\mathcal P}
\newcounter{Hilfssatz}
\newcounter{Definition}
\newcounter{Beispiel}
\newcounter{Satz}
\newcounter{Algorithmus}



\setlength{\parindent}{0pt}
\makeatletter
  \@addtoreset{Definition}{subsection}
\makeatother
\newenvironment{Definition}{
\bigskip
        
        \setlength{\parindent}{0pt}
        \addtocounter{Definition}{1}
        \textbf{\textsf{Definition \thesubsection.\theDefinition}:}\\}{
        \nopagebreak
        \vspace{-1.0ex}
        \bigskip
        
}
\makeatletter
  \@addtoreset{Hilfssatz}{subsection}
\makeatother
\newenvironment{Hilfssatz}{
\medskip
        
        \setlength{\parindent}{0pt}
        \addtocounter{Hilfssatz}{1}
        \textbf{\textsf{Hilfssatz \thesubsection.\theHilfssatz}:}\\}{
        \nopagebreak
        \vspace{-1.0ex}
        \bigskip\\
        
}

\makeatletter
  \@addtoreset{Satz}{subsection}
\makeatother
\newenvironment{Satz}{
\medskip
        
        \setlength{\parindent}{0pt}
        \addtocounter{Satz}{1}
        \textbf{\textsf{Satz \thesubsection.\theSatz}:}\\}{
        \nopagebreak
        \vspace{-1.0ex}
        \bigskip\\
        
}

\makeatletter
  \@addtoreset{Beispiel}{subsection}
\makeatother
\newenvironment{Beispiel}{
\medskip
        
        \setlength{\parindent}{0pt}
        \addtocounter{Beispiel}{1}
        \textbf{\textsf{Beispiel \thesubsection.\theBeispiel}:}\\}{
        \nopagebreak
        \vspace{-1.0ex}
        \bigskip
        
}

\setlength{\parindent}{0pt}
\newenvironment{proof}{
\bigskip
        
        \setlength{\parindent}{0pt}
        \textbf{Beweis:}\\}{
        \nopagebreak
        \vspace{-1.0ex}
        \begin{flushright}
             $\square$
        \end{flushright}
        \bigskip
        
}

\makeatletter
  \@addtoreset{Algorithmus}{subsection}
\makeatother
\newenvironment{Algorithmus}{
\medskip
        
        \setlength{\parindent}{0pt}
        \addtocounter{Algorithmus}{1}
        \textbf{\textsf{Algorithmus \thesubsection.\theAlgorithmus}:}}{
        \nopagebreak
        \vspace{-1.0ex}
        \bigskip
        
}

\usepackage{lastpage}% F\"ur die Verweise innerhalb des  Symbolverzeichnisses

\usepackage{nomencl} % Symbolverzeichnis
\let\symb\nomenclature %% Es genuegt \symb statt \nomenclature zu  schreiben
\setlength{\nomlabelwidth}{.25\hsize}
\renewcommand{\nomlabel}[1]{#1 \dotfill}
\setlength{\nomitemsep}{-\parsep}\renewcommand{\nomname}
{Symbolverzeichnis}

\setlength{\nomitemsep}{-\parsep}
\usepackage{array} %notwendig um neue Spaltentypen zu definieren
\newcolumntype{B}[1]{>{\centering\arraybackslash}m{#1}}
\makenomenclature



%Bsp für reelle Zahlen
\begin{document}
\begin{titlepage}
\thispagestyle{empty} \enlargethispage{1.4in}

\begin{center}

\rule[1ex]{157.5mm}{0.5mm}

\LARGE\bf Hochschule für angewandte Wissenschaften Coburg\\

\vfill

\rm Institut für Informatik


\begin{figure}
	    \centering
				     \includegraphics[width=0.5\textwidth]{Logo.png}
\end{figure}

\vfill

\Huge \bf Projektbericht Datamining

\vfill

\normalsize Michael Krasser

\vfill

Betreuer:  Dr.~Detlef~Bittner

Abgabe des Berichts: 22 Januar 2019

\vfill

Coburg, \today

\rule[-1ex]{157.5mm}{0.5mm}

\vfill

\end{center}

\end{titlepage}

\newpage
\tableofcontents
\pagebreak
\section{Geschäftsverständnis}
\subsection{Beschreibung der Situation}
Das zu untersuchende System beschreibt einen Onlineshop dessen Produktpalette aus Medien wie z.B. CDs, Bücher, Hörbücher, ebooks und ebook-Readern besteht. Um gegenüber großen Online-Anbietern wettbewerbsfähig zu sein muß der Onlineshop regelmäßig Maßnahmen zur Aquise und Kundenbindung ergreifen. Eine Möglichkeit zur Kundenbindung besteht in einer Gutscheinausstellung. 

\subsubsection{Problembeschreibung}
Viele Kunden tätigen meist nur eine Bestellung im Onlineshop. Es kann keine Vorhersage darüber getroffen werden ob ein Kunde eine weitere Bestellung tätigen oder sein Interesse an den Produkten des Onlineshops erlischt. Das Ziel dieser Arbeit besteht in der Bestimmung der Loyalität des einzelnen Kunden anhand von Merkmalen die sich aus den beigelegten Datensätzen erschließen lassen. Ist die Wahrscheinlichkeit sehr gering, dass ein Kunde einen weiteren Einkauf tätigt, erscheint es sinnvoll diesen mittels eines Gutscheins an den Onlineshop zu erinnern. Ist dagegen die Wahrscheinlicht eines erneuten Einkaufs eher hoch, so verursacht die Zusendung eines Gutscheins nur unnötige Kosten.  
\par
Um dieses Problem zu lösen, soll mittels des CRISP-DM [IBM-Consortium, 2012] Modells
und des Datamining-Tools SPSS von IBM eine Bewertung durchgeführt werden, welche die
Loyalität des Kunden bestimmen soll. Hierfür wurden zwei *.txt-Dateien über Kundendaten geliefert.
Als Lösung wird eine Abbildung der Kundennummer auf das zu erwartende Kaufverhalten gefordert. 

\subsection{Situationsbewertung}

\subsubsection{Beschreibung der gelieferten Daten}
Bei den zur Verfügung gestellten Daten  handelt es sich um Auszüge aus den Bestelldaten einer Kundendatenbank.
Es gibt Spalten für das Datum einer Lieferung, Accounterstellung und der
ersten Bestellung, Domain und Kundennummer des Kunden und eine Reihe von Flags, die das Kaufverhalten des Kunden beschreiben:
Es wird registriert ob gebrauchte oder importierte Artikel gekauft wurden, wie die Ware zum Kunden gelangt ist, wie der Kunde bezahlt hat, ob die Lieferung zurückgeschickt oder gecancelt wurde und wie viele Artikel der Kunde auf einmal gekauft hat.

\subsubsection{Risikofaktoren}

Das Risiko des Onlineshops liegt in der Versendung zu vieler Gutscheine. Diese Gutscheine können dann auch Kunden erreichen, die auch ohne diese wieder eine Bestellung getätigt hätten. In diesem Fall verliert der Shop pro Bestellung 5 \euro. Je nachdem, wie viele der Gutscheine unnötig ausgegeben werden, kann
sich diese Vorgehensweise als unrentabel erweisen. (Einfluß des falsch positiven und des falsch negativen Fehlers bleiben unberücksichtig.)

\subsubsection{Erfolgskriterien}
Es konnten drei zentrale Erfolgskriterien für das Projekt identifiziert werden:
\begin{itemize}
	\item Entscheidung ob ein Kunde einen Gutschein erhält
	\item Maximierung des Gewinns im Datensatz dmc2010 class.txt
	\item Ab einem Gewinn von 9.310,50\euro\; im Testdatensatz und 8547,50\euro\; im class-Datensatz liegt ein erfolgreiches Datamining vor
\end{itemize}

\textbf{Bemerkung}:
\par
Zur Berechnung des Gewinnes (bzw. des Verlustes) wurde folgende Entscheidungsmatrix vorgeschrieben:
\begin{center}
\begin{tabular}{|c | c | c |}
\hline
 & Kein Wiederkäufer & Wiederkäufer
\\
\hline
Kein Gutschein & 0 & 0
\\
\hline
Gutschein & 1.5 & -5
\\
\hline
\end{tabular}
\end{center}
Jeder Gutschein, der an eine Person ausgegeben wurde, die ansonsten nicht wieder bestellen
würde, ergibt einen Gewinn von 1,50\euro\; für den Onlineshop. Jeder Kunde der fälschlicherweise
einen Gutschein bekommt, bedeutet einen Verlust in  Höhe des Wertes des Gutscheins, also 5\euro.

Als Referenzwert\label{Referenzwerte} für die Modellbildung wird folgende Situation
zugrunde gelegt: Jedem Kunden wird ein Gutschein zugesand.Unter der Annahme, dass jeder dieser Kunden auch wieder eine Bestellung tätigt, erhält man im Testdatensatz (26377 Kunden tätigten keine weitere Bestellung innerhalb von 90 Tagen, 6051 aber schon) einen Verlust von
\begin{align*}
(26377 *1.50 - 6051*5)\text{\euro} = 9310,50 \text{\euro}
\end{align*}
Diese Werte müssen in der Modellbildung übertroffen werden.

\subsection{Projektplanung}
Für die Projektdurchführung wurde zu Projektbeginn der Aufwand in
Wochen für die jeweiligen Projektschritte des CRISP-DM Modells geschätzt:
\begin{itemize}
	\item Geschäftsverständnis: ca eine Woche
	\item Datenverständnis: ca eine Woche
	\item Datenvorbereitung: ca zweieinhalb Wochen
	\item Modellbildung: ca zwei Wochen
	\item Evaluierung: ca eine halbe Woche
	\item Bereitstellung: ca eine halbe Woche
\end{itemize}

\section{Datenverständnis}
\subsection{Datensammlung}
Einige Attribute des Datensatzes sind weniger erfolgsversprechend als andere. Da bereits eine
Beschreibung der einzelnen Attribute zugeliefert wurde, kann an dieser Stelle eine Vorsortierung
der Spalten erfolgen. Um versehentliche Korrelationen mit unwichtigen Daten im Ergebnis der
Untersuchung zu vermeiden, sollten diese vor der Betrachtung aussortiert werden.



\subsubsection{Vernachlässigbare Felder}
Unwichtige Felder lassen sich anhand verschiedener Merkmale identifizieren. Bei einer
auffallend schlechten Datenqualität sollte das Feld aussortiert werden(fehlende Einträge, Ausreisser).
Außerdem existieren Felder, anhand deren Beschreibung bereits erkannt werden
kann, dass diese keinen Einfluß auf das Ergebnis haben können. Als solche Felder wurdendie im Folgenden beschriebenen identifiziert:
\par
\vspace{1cm}
\begin{minipage}[h]{.5\textwidth}
\begin{itemize}
	\item "`salutation"'
	\item "`domain"'
	\item "`model"'
	\item "`invoicepostcode"'
	\item "`delivpostcode"'
\end{itemize}
\end{minipage}
\hfill
\begin{minipage}[h]{.5\textwidth}
\begin{itemize}
  \item "`advertisingdatacode"'
	\item "`points"'
	\item "`shippingcosts"'
	\item "`weight"'
	\item "`used"'
	\end{itemize}
\end{minipage}
\par
\vspace{1cm}
Die Anrede ("`salutation"') wurde zu Beginn als relevant eingestuft, in einem späteren Analyseschritt aber entfernt:
Email-Domain oder Anrede  eines Kunden haben sicher wenig bis gar keinen Einfluß auf das Kaufverhalten des Kunden. Die Bedeutung des Feldes "`model"' konnte nicht schlüssig geklärt werden, daher wurde es auf Grund des vernachlässigbaren Prädikatoreinflusses entfernt.
Die Rechnungsadresse ("`invoicepostcode"') und die Lieferadresse ("`delivpostcode"') wurden auch verworfen da hier keine Rückschlüsse auf das Kaufverhalten des Kunden erkennbar sind. Außerdem erwies sich die Datenqualität des Feldes "`invoicepostcode"' als schlecht wegen vieler NULL-Werte.
Der Werbecode ( "`advertisingdatacode"') wurde auf Grund schlechter Datenqualität entfernt. Punkte eingelöst ("`points"') schien zunächst relevant, da es Aufschluss über die Empfänglichkeit des Kunden für Werbeaktionen gibt, erwies sich aber als nicht hilfreich und wurde in einem späteren Schritt auf Grund der der zu schlechtene Datenqualität entfernt. Spezifische Artikelinformation wir Versankosten ("`shippingcosts"') Gewicht ("`weight"') und Second Hand ("`used"') wurden auch entfernt, da sich hieraus keinerlei Information ableiten ließ.

Zusätzlich wurden die Produktkategorien w0 - w10 gleich zu Beginn als uniteressant für die Modellbildung eingestuft.


\subsubsection{Erfolgsversprechende Felder}
Bewertet wurde die Datenqualität, die Datenplausibilität und der
Vorteil, den die Daten für die Auswertung ergeben könnten. Das Feld "`datecreated"' kann in Verbindung
mit dem Feld "`date"' dazu genutzt werden, um die Zeit zwischen der Accounterstellung
und der ersten Bestellung zu berechnen. Daher werden beide Attribute beibehalten. Auch aus
den Feldern "`cancel"', "`deliverydatepromised"', "`deliverydatereal"'
etc. können zusammengesetzte Felder ermittelt werden.
Zusätzlich sollten die Spalten "`gift"', "`voucher"' und "`newsletter"' für die Auswertung verwendet werden, da sie Aufschluss über das Kaufverhalten des Kunden geben: Durch das Feld
"`newsletter"' kann beispielsweise erkannt werden, ob der Kunde generell über Interesse an den Produkten
des Shops verfügt. 

\subsection{Datenbeschreibung}

\subsubsection{Datenmenge}
Im Trainingsdatensatz und dem Vorhersagedatensatz befinden sich 32.428 bzw. 32.427 Datensätze.
Datenmengen dieser Größe sind für eine vollständige Analyse im Sinne von Big Data nicht
ausreichend, was sich auch in der späteren Modellierung der Daten zeigte.

\subsubsection{Untersuchung der Daten}

Jede Zeile der Dateien liefert Informationen über eine Bestellung. Jede Bestellung lässt sich eindeutig einem Kunden mittels des
Feldes "`customernumber"' zuordnen. Anhand der angegebenen Werte der Bestellung lässt sich
feststellen, ob die versprochene Lieferzeit eingehalten wurde, welche Artikel bestellt wurden
und wie sich diese zusammensetzen. Weitere Daten sind indirekt enthalten und lassen sich aus vorhandenen Daten ermitteln.

\subsubsection{Konsistenz}
Felder mit schlechter Datenqualität (fehlende Werte, NULL-Werte, viele Ausreisser,\ldots) sollten unter Verwendung eines Filters entfernt werden.
Hierzu gehören die Felder "` points "`und "`advertisingdatacode"' (viele fehlende Werte), "`invoicepostcode"' und "`deliverydatereal"' (NULL-Werte).
Jedoch mussten diese beiden Felder zunächst für die Datenvorbereitung zur Erzeugung neuer Attribute beibehalten werden.

\section{Vorbereitung der Daten}
\subsection{Bereinigung der Daten}
Die Bereinigung der Daten erfolgte in zwei Schritten: {\color{red}{wie wurde aussortiert??}}
\par
Zunächst wurden alle NULL-Werte aussortiert und danach Extremwerte (Ausreißer).
Die Wertemenge einiger Felder enthielt viele NULL-Werte welche bei der späteren Modellierung hinderlich sein könnten.
Entsprechend wurden diese Werte (z.B bei "`deliverydatereal"' und "`deliverydatepromised"') mit einem Auswahl-Knoten des SPSS Modelers entfernt, 
was eine deutliche Reduzierung der Datenmenge nach sich zog: Etwa 7000 Datensätze  wurden entfernt.
Unrealistische Werte wurden ebenfalls mit dem Auswahl-Knoten entfernt.

\subsection{Erstellung neuer Felder}
Durch studieren der Situation konnten neue Felder aus bestehenden Feldern gebildet werden:  
\begin{framed}
\begin{verbatim}
diffPromReal = deliverydatepromised - deliverydatereal
\end{verbatim}
\end{framed}
Differenz aus versprochenem und tatsächlichem Lieferdatum: Liegt das tatsächliche Lieferdatum nach dem versprochenen ist die Wahrscheinlichkeit recht groß den Kunden zu verlieren. 
\begin{framed}
\begin{verbatim}
timeToFirstOrder = datecreated - date
\end{verbatim}
\end{framed}
Zeitdauer zwischen Registrierung des Kunden und seiner ersten Bestellung.
\begin{framed}
\begin{verbatim}
gotAdvertised = newsletter OR voucher
\end{verbatim}
\end{framed}
Aufschluss ob Kunde an Werbemaßnahmen teilnimmt.
\begin{framed}
\begin{verbatim}
giftOnlyBuyer = gift AND (numberitems == 1) 
\end{verbatim}
\end{framed}
Aufschluss ob Kunde ein Geschenk bestellt hat.
\begin{framed}
\begin{verbatim}
isRealGiftOnlyBuyer = (giftOnlyBuyer == 1) AND (date == datecreated)
\end{verbatim}
\end{framed}
Aufschluss ob Kunde sich nur für die Bestellung eines Geschenks registriert hat.
\begin{framed}
\begin{verbatim}
itemsKept = numberitems - remi - cancel 
\end{verbatim}
\end{framed}
Aufschluss  über die Anzahl der Artikel die der Kunde bestelt und auch behalten hat.
\subsection{Bewertung der Felder}

Nachdem neue Felder erstellt wurden, muss der Einfluss dieser auf die Modellierung bewertet werden.
Hierfür wurde nach der Datenvorbereitung  ein Knoten zur Merkmalauswahl angefügt.

Dabei ergab sich, dass 10 der 19 Felder als bedeutsam (100 \%) eingestuft wurden, ein weiteres Feld besaß eine Korrelation von über 90 \%,
die restlichen wurden als unbedeutsam (Korrelation von unter 90\%) eingestuft.
Korrelationswerte sollten lediglich als Richtlinie betrachtet werden, da die Gewichtung
vieler Attribute bei 100\% lag.  Unter den 10 bedeutsamen Feldern befanden sich auch einige der
generierten:

\vspace{0.2cm}
\par
	\begin{minipage}[h]{.5\textwidth}
	\begin{center}
	"`itemsKept"'
	\end{center}
	\end{minipage}
	\hfill
	\begin{minipage}[h]{.5\textwidth}
	\begin{center}
	"`gotAdvertised"'
	\end{center}
	\end{minipage}
	\vspace{0.2cm}
\par
Damit stehen für die Modellierung zehn Felder zu Verfügung.
\section{Modellierung}
Als Grundlage für die Modellierung werden die unter ~\ref{Referenzwerte} ermittelten Referenzwerte verwendet. Ziel der Modellierung ist eine Maximierung des Gewinns, also ein Modell zu entwickeln, dessen Gewinn die berechneten Gewinne übertrifft.  
\subsection{Wahl der Modellierungsknoten}
Hinsichtlich der Struktur der Aufgabe kann man sich auf Baummodelle bei der Wahl der Knoten im SPSS Modeler beschränken. Hierfür wurden folgende Modellierungsknoten in Betracht gezogen:
\begin{itemize}
	\item Random-Trees
	\item XRandom-Tree
	\item CHAID
	\item Tree-AS
	\item C\&R-Tree
\end{itemize} 
Hinsichtlich der Validierung wurde zusätzlich überprüft, ob sich die getroffene Merkmalauswahl positiv auf den erzielten Gewinn auswirken konnte. Dabei konnte erkannt werden, welche Felder welchen Einfluß auf die Modellierung besitzen.
\subsection{Test-Design}
\subsubsection{Random-Trees}
Dieser Knoten erwies sich als bester Moellierungsknoten.
 Der Algorithmus entwickelt ein Modell, welches sich aus verschiedenen Entscheidungsbäumen zusammensetzt.
 Random Trees entspricht in der grundlegenden Vorghensweise der des C\&R-Baums, erweitert
diesen jedoch um zwei Punkte: Zum einen wird in diesem Modell Bagging verwendet, um
ein Overfitting des Datensatzes zu vermeiden. Zum anderen wird für jede Aufteilung des Baums
lediglich eine Stichprobe der Input-Werte zur Errechnung der Unreinheit benutzt. [IBM, 2016,
S.104] Mittels diesem Modells konnten auf dem Trainingsdatensatz 11.495 e Gewinn erreicht
werden.
\subsubsection{XRandom-Tree}
\subsubsection{CHAID}
Der CHAID-Knoten generiert Entscheidungsbäume unter der Verwendung der $\chi^2$  - Verteilung.
Im Gegensatz zum C\&R-Baum können mit diesem Modell auch Bäume mit mehr als zwei
Verzweigungen generiert werden. Berücksichtigt wurde das Verhalten des Gewinnes unter Merkmalauswahl und Boosting. 

\begin{center}
\begin{tabular}{ c | c | c }
 & Boosting & $\overline{\text{Boosting}}$
\\
\hline
Merkwahlauswahl  &  8,548.50\euro & 8,575.00\euro
\\
$\overline{\text{Merkwahlauswahl}}$ & 8,697.00\euro  &  8,548.50\euro
\\
\end{tabular}
\end{center}
\subsubsection{Tree-AS}
\subsubsection{C\&R-Tree}

\newpage
\begin{titlepage}
\section*{Erklärung:}
\noindent {\large Die vorliegende Projektarbeit wurde am Institut für Informatik  der Hochschule Coburg nach einem Thema von Herrn Dr.~Detlef~Bittner erstellt.
\newline Hiermit versichere ich, dass ich diese Arbeit selbstständig angefertigt und dazu nur die angegebenen Quellen verwendet habe.

\vspace{1.5cm}

\noindent Coburg, den \today

\vspace{0.5cm}

\raggedleft Michael Krasser\quad\quad \par}

\vfill
\end{titlepage}
\end{document}
